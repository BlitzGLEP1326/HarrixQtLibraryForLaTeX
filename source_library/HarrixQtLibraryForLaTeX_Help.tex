\documentclass[a4paper,12pt]{article}

%%% HarrixLaTeXDocumentTemplate
%%% Версия 1.17
%%% Шаблон документов в LaTeX на русском языке. Данный шаблон применяется в проектах HarrixTestFunctions, MathHarrixLibrary, Standard-Genetic-Algorithm  и др.
%%% https://github.com/Harrix/HarrixLaTeXDocumentTemplate
%%% Шаблон распространяется по лицензии Apache License, Version 2.0.

%%% Поля и разметка страницы %%%
\usepackage{lscape} % Для включения альбомных страниц
\usepackage{geometry} % Для последующего задания полей

%%% Кодировки и шрифты %%%
\usepackage{pscyr} % Нормальные шрифты
\usepackage{cmap} % Улучшенный поиск русских слов в полученном pdf-файле
\usepackage[T2A]{fontenc} % Поддержка русских букв
\usepackage[utf8]{inputenc} % Кодировка utf8
\usepackage[english, russian]{babel} % Языки: русский, английский

%%% Математические пакеты %%%
\usepackage{amsthm,amsfonts,amsmath,amssymb,amscd} % Математические дополнения от AMS
%%% Для жиного курсива в формулах %%%
\usepackage{bm}

%%% Оформление абзацев %%%
\usepackage{indentfirst} % Красная строка
\usepackage{setspace} % Расстояние между строками
\usepackage{enumitem} % Для список обнуление расстояния до абзаца

%%% Цвета %%%
\usepackage[usenames]{color}
\usepackage{color}
\usepackage{colortbl}

%%% Таблицы %%%
\usepackage{longtable} % Длинные таблицы
\usepackage{multirow,makecell,array} % Улучшенное форматирование таблиц

%%% Общее форматирование
\usepackage[singlelinecheck=off,center]{caption} % Многострочные подписи
\usepackage{soul} % Поддержка переносоустойчивых подчёркиваний и зачёркиваний

%%% Библиография %%%
\usepackage{cite}

%%% Гиперссылки %%%
\usepackage{hyperref}

%%% Изображения %%%
\usepackage{graphicx} % Подключаем пакет работы с графикой
\usepackage{epstopdf}
\usepackage{subcaption}

%%% Отображение кода %%%
\usepackage{xcolor}
\usepackage{listings}
\usepackage{caption}

%%% Псевдокоды %%%
\usepackage{algorithm} 
\usepackage{algpseudocode}

%%% Рисование графиков %%%
\usepackage{pgfplots}

%%% HarrixLaTeXDocumentTemplate
%%% Версия 1.22
%%% Шаблон документов в LaTeX на русском языке. Данный шаблон применяется в проектах HarrixTestFunctions, MathHarrixLibrary, Standard-Genetic-Algorithm  и др.
%%% https://github.com/Harrix/HarrixLaTeXDocumentTemplate
%%% Шаблон распространяется по лицензии Apache License, Version 2.0.

%%% Макет страницы %%%
% Выставляем значения полей (ГОСТ 7.0.11-2011, 5.3.7)
\geometry{a4paper,top=2cm,bottom=2cm,left=2.5cm,right=1cm}

%%% Выравнивание и переносы %%%
\sloppy % Избавляемся от переполнений
\clubpenalty=10000 % Запрещаем разрыв страницы после первой строки абзаца
\widowpenalty=10000 % Запрещаем разрыв страницы после последней строки абзаца


%%% Библиография %%%
\makeatletter
\bibliographystyle{utf8gost71u}  % Оформляем библиографию по ГОСТ 7.1 (ГОСТ Р 7.0.11-2011, 5.6.7)
\renewcommand{\@biblabel}[1]{#1.} % Заменяем библиографию с квадратных скобок на точку
\makeatother

%%% Изображения %%%
\graphicspath{{images/}} % Пути к изображениям
% Поменять двоеточние на точку в подписях к рисунку
\RequirePackage{caption}
\DeclareCaptionLabelSeparator{defffis}{. }
\captionsetup{justification=centering,labelsep=defffis}

%%% Абзацы %%%
% Отсупы между строками
\singlespacing
\setlength{\parskip}{0.3cm} % отступы между абзацами
\linespread{1.3} % Полуторный интвервал (ГОСТ Р 7.0.11-2011, 5.3.6)

% Оформление списков
\setlist{leftmargin=1.5cm,topsep=0pt}
% Используем дефис для ненумерованных списков (ГОСТ 2.105-95, 4.1.7)
\renewcommand{\labelitemi}{\normalfont\bfseries{--}}

%%% Цвета %%%
% Цвета для кода
\definecolor{string}{HTML}{B40000} % цвет строк в коде
\definecolor{comment}{HTML}{008000} % цвет комментариев в коде
\definecolor{keyword}{HTML}{1A00FF} % цвет ключевых слов в коде
\definecolor{morecomment}{HTML}{8000FF} % цвет include и других элементов в коде
\definecolor{сaptiontext}{HTML}{FFFFFF} % цвет текста заголовка в коде
\definecolor{сaptionbk}{HTML}{999999} % цвет фона заголовка в коде
\definecolor{bk}{HTML}{FFFFFF} % цвет фона в коде
\definecolor{frame}{HTML}{999999} % цвет рамки в коде
\definecolor{brackets}{HTML}{B40000} % цвет скобок в коде
% Цвета для гиперссылок
\definecolor{linkcolor}{HTML}{799B03} % цвет ссылок
\definecolor{urlcolor}{HTML}{799B03} % цвет гиперссылок
\definecolor{citecolor}{HTML}{799B03} % цвет гиперссылок
\definecolor{gray}{rgb}{0.4,0.4,0.4}
\definecolor{tableheadcolor}{HTML}{E5E5E5} % цвет шапки в таблицах
\definecolor{darkblue}{rgb}{0.0,0.0,0.6}
% Цвета для графиков
\definecolor{plotcoordinate}{HTML}{88969C}% цвет точек на координатых осях (минимум и максимум)
\definecolor{plotgrid}{HTML}{ECECEC} % цвет сетки
\definecolor{plotmain}{HTML}{97BBCD} % цвет основного графика
\definecolor{plotsecond}{HTML}{FF0000} % цвет второго графика, если графика только два
\definecolor{plotsecondgray}{HTML}{CCCCCC} % цвет второго графика, если графика только два. В сером виде.
\definecolor{darkgreen}{HTML}{799B03} % цвет темно-зеленого

%%% Отображение кода %%%
% Настройки отображения кода
\lstset{
language=C++, % Язык кода по умолчанию
morekeywords={*,...}, % если хотите добавить ключевые слова, то добавляйте
% Цвета
keywordstyle=\color{keyword}\ttfamily\bfseries,
%stringstyle=\color{string}\ttfamily,
stringstyle=\ttfamily\color{red!50!brown},
commentstyle=\color{comment}\ttfamily\itshape,
morecomment=[l][\color{morecomment}]{\#}, 
% Настройки отображения     
breaklines=true, % Перенос длинных строк
basicstyle=\ttfamily\footnotesize, % Шрифт для отображения кода
backgroundcolor=\color{bk}, % Цвет фона кода
frame=lrb,xleftmargin=\fboxsep,xrightmargin=-\fboxsep, % Рамка, подогнанная к заголовку
rulecolor=\color{frame}, % Цвет рамки
tabsize=3, % Размер табуляции в пробелах
% Настройка отображения номеров строк. Если не нужно, то удалите весь блок
%numbers=left, % Слева отображаются номера строк
%stepnumber=1, % Каждую строку нумеровать
%numbersep=5pt, % Отступ от кода 
%numberstyle=\small\color{black}, % Стиль написания номеров строк
% Для отображения русского языка
extendedchars=true,
literate={Ö}{{\"O}}1
  {Ä}{{\"A}}1
  {Ü}{{\"U}}1
  {ß}{{\ss}}1
  {ü}{{\"u}}1
  {ä}{{\"a}}1
  {ö}{{\"o}}1
  {~}{{\textasciitilde}}1
  {а}{{\selectfont\char224}}1
  {б}{{\selectfont\char225}}1
  {в}{{\selectfont\char226}}1
  {г}{{\selectfont\char227}}1
  {д}{{\selectfont\char228}}1
  {е}{{\selectfont\char229}}1
  {ё}{{\"e}}1
  {ж}{{\selectfont\char230}}1
  {з}{{\selectfont\char231}}1
  {и}{{\selectfont\char232}}1
  {й}{{\selectfont\char233}}1
  {к}{{\selectfont\char234}}1
  {л}{{\selectfont\char235}}1
  {м}{{\selectfont\char236}}1
  {н}{{\selectfont\char237}}1
  {о}{{\selectfont\char238}}1
  {п}{{\selectfont\char239}}1
  {р}{{\selectfont\char240}}1
  {с}{{\selectfont\char241}}1
  {т}{{\selectfont\char242}}1
  {у}{{\selectfont\char243}}1
  {ф}{{\selectfont\char244}}1
  {х}{{\selectfont\char245}}1
  {ц}{{\selectfont\char246}}1
  {ч}{{\selectfont\char247}}1
  {ш}{{\selectfont\char248}}1
  {щ}{{\selectfont\char249}}1
  {ъ}{{\selectfont\char250}}1
  {ы}{{\selectfont\char251}}1
  {ь}{{\selectfont\char252}}1
  {э}{{\selectfont\char253}}1
  {ю}{{\selectfont\char254}}1
  {я}{{\selectfont\char255}}1
  {А}{{\selectfont\char192}}1
  {Б}{{\selectfont\char193}}1
  {В}{{\selectfont\char194}}1
  {Г}{{\selectfont\char195}}1
  {Д}{{\selectfont\char196}}1
  {Е}{{\selectfont\char197}}1
  {Ё}{{\"E}}1
  {Ж}{{\selectfont\char198}}1
  {З}{{\selectfont\char199}}1
  {И}{{\selectfont\char200}}1
  {Й}{{\selectfont\char201}}1
  {К}{{\selectfont\char202}}1
  {Л}{{\selectfont\char203}}1
  {М}{{\selectfont\char204}}1
  {Н}{{\selectfont\char205}}1
  {О}{{\selectfont\char206}}1
  {П}{{\selectfont\char207}}1
  {Р}{{\selectfont\char208}}1
  {С}{{\selectfont\char209}}1
  {Т}{{\selectfont\char210}}1
  {У}{{\selectfont\char211}}1
  {Ф}{{\selectfont\char212}}1
  {Х}{{\selectfont\char213}}1
  {Ц}{{\selectfont\char214}}1
  {Ч}{{\selectfont\char215}}1
  {Ш}{{\selectfont\char216}}1
  {Щ}{{\selectfont\char217}}1
  {Ъ}{{\selectfont\char218}}1
  {Ы}{{\selectfont\char219}}1
  {Ь}{{\selectfont\char220}}1
  {Э}{{\selectfont\char221}}1
  {Ю}{{\selectfont\char222}}1
  {Я}{{\selectfont\char223}}1
  {і}{{\selectfont\char105}}1
  {ї}{{\selectfont\char168}}1
  {є}{{\selectfont\char185}}1
  {ґ}{{\selectfont\char160}}1
  {І}{{\selectfont\char73}}1
  {Ї}{{\selectfont\char136}}1
  {Є}{{\selectfont\char153}}1
  {Ґ}{{\selectfont\char128}}1
  {\{}{{{\color{brackets}\{}}}1 % Цвет скобок {
  {\}}{{{\color{brackets}\}}}}1 % Цвет скобок }
}
% Для настройки заголовка кода
\DeclareCaptionFont{white}{\color{сaptiontext}}
\DeclareCaptionFormat{listing}{\parbox{\linewidth}{\colorbox{сaptionbk}{\parbox{\linewidth}{#1#2#3}}\vskip-4pt}}
\captionsetup[lstlisting]{format=listing,labelfont=white,textfont=white}
\renewcommand{\lstlistingname}{Код} % Переименование Listings в нужное именование структуры
% Для отображения кода формата xml
\lstdefinelanguage{XML}
{
  morestring=[s]{"}{"},
  morecomment=[s]{?}{?},
  morecomment=[s]{!--}{--},
  commentstyle=\color{comment},
  moredelim=[s][\color{black}]{>}{<},
  moredelim=[s][\color{red}]{\ }{=},
  stringstyle=\color{string},
  identifierstyle=\color{keyword}
}

%%% Гиперссылки %%%
\hypersetup{pdfstartview=FitH,  linkcolor=linkcolor,urlcolor=urlcolor,citecolor=citecolor, colorlinks=true}

%%% Псевдокоды %%%
% Добавляем свои блоки
\makeatletter
\algblock[ALGORITHMBLOCK]{BeginAlgorithm}{EndAlgorithm}
\algblock[BLOCK]{BeginBlock}{EndBlock}
\makeatother

% Нумерация блоков
\usepackage{caption}% http://ctan.org/pkg/caption
\captionsetup[ruled]{labelsep=period}
\makeatletter
\@addtoreset{algorithm}{chapter}% algorithm counter resets every chapter
\makeatother
\renewcommand{\thealgorithm}{\thechapter.\arabic{algorithm}}% Algorithm # is <chapter>.<algorithm>

%%% Формулы %%%
%Дублирование символа при переносе
\newcommand{\hmm}[1]{#1\nobreak\discretionary{}{\hbox{\ensuremath{#1}}}{}}

%%% Таблицы %%%
% Раздвигаем в таблице без границ отступы между строками в новой команде
\newenvironment{tabularwide}%
{\setlength{\extrarowheight}{0.3cm}\begin{tabular}{  p{\dimexpr 0.45\linewidth-2\tabcolsep} p{\dimexpr 0.55\linewidth-2\tabcolsep}  }}  {\end{tabular}}
\newenvironment{tabularwide08}%
{\setlength{\extrarowheight}{0.3cm}\begin{tabular}{  p{\dimexpr 0.8\linewidth-2\tabcolsep} p{\dimexpr 0.2\linewidth-2\tabcolsep}  }}  {\end{tabular}}

% Многострочная ячейка в таблице
\newcommand{\specialcell}[2][c]{%
  {\renewcommand{\arraystretch}{1}\begin{tabular}[t]{@{}l@{}}#2\end{tabular}}}

% Многострочная ячейка, где текст не может выйти за границы
\newcolumntype{P}[1]{>{\raggedright\arraybackslash}p{#1}}
\newcommand{\specialcelltwoin}[2][c]{%
  {\renewcommand{\arraystretch}{1}\begin{tabular}[t]{@{}P{1.98in}@{}}#2\end{tabular}}}
  
% Команда для переворачивания текста в ячейке таблицы на 90 градусов
\newcommand*\rot{\rotatebox{90}}

%%% Рисование графиков %%%
\pgfplotsset{
every axis legend/.append style={at={(0.5,-0.13)},anchor=north,legend cell align=left},
tick label style={font=\tiny\scriptsize},
label style={font=\scriptsize},
legend style={font=\scriptsize},
grid=both,
minor tick num=2,
major grid style={plotgrid},
minor grid style={plotgrid},
axis lines=left,
legend style={draw=none},
/pgf/number format/.cd,
1000 sep={}
}
% Карта цвета для трехмерных графиков в стиле графиков Mathcad
\pgfplotsset{
/pgfplots/colormap={mathcad}{rgb255(0cm)=(76,0,128) rgb255(2cm)=(0,14,147) rgb255(4cm)=(0,173,171) rgb255(6cm)=(32,205,0) rgb255(8cm)=(229,222,0) rgb255(10cm)=(255,13,0)}
}
% Карта цвета для трехмерных графиков в стиле графиков Matlab
\pgfplotsset{
/pgfplots/colormap={matlab}{rgb255(0cm)=(0,0,128) rgb255(1cm)=(0,0,255) rgb255(3cm)=(0,255,255) rgb255(5cm)=(255,255,0) rgb255(7cm)=(255,0,0) rgb255(8cm)=(128,0,0)}
}

%%% Разное %%%
% Галочки для отмечания в тескте вариантов как OK
\def\checkmark{\tikz\fill[black,scale=0.3](0,.35) -- (.25,0) -- (1,.7) -- (.25,.15) -- cycle;}
\def\checkmarkgreen{\tikz\fill[darkgreen,scale=0.3](0,.35) -- (.25,0) -- (1,.7) -- (.25,.15) -- cycle;} 
\def\checkmarkred{\tikz\fill[red,scale=0.3](0,.35) -- (.25,0) -- (1,.7) -- (.25,.15) -- cycle;}
\def\checkmarkbig{\tikz\fill[black,scale=0.5](0,.35) -- (.25,0) -- (1,.7) -- (.25,.15) -- cycle;}
\def\checkmarkbiggreen{\tikz\fill[darkgreen,scale=0.5](0,.35) -- (.25,0) -- (1,.7) -- (.25,.15) -- cycle;} 
\def\checkmarkbigred{\tikz\fill[red,scale=0.5](0,.35) -- (.25,0) -- (1,.7) -- (.25,.15) -- cycle;}

%% Следующие блоки расскоментировать при необходимости

%%% Кодировки и шрифты %%%
%\ifxetex
%\setmainlanguage{russian}
%\setotherlanguage{english}
%\defaultfontfeatures{Ligatures=TeX,Mapping=tex-text}
%\setmainfont{Times New Roman}
%\newfontfamily\cyrillicfont{Times New Roman}
%\setsansfont{Arial}
%\newfontfamily\cyrillicfontsf{Arial}
%\setmonofont{Courier New}
%\newfontfamily\cyrillicfonttt{Courier New}
%\else
%\IfFileExists{pscyr.sty}{\renewcommand{\rmdefault}{ftm}}{}
%\fi

%%% Колонтитулы %%%
% Порядковый номер страницы печатают на середине верхнего поля страницы (ГОСТ Р 7.0.11-2011, 5.3.8)
%\makeatletter
%\let\ps@plain\ps@fancy              % Подчиняем первые страницы каждой главы общим правилам
%\makeatother
%\pagestyle{fancy}                   % Меняем стиль оформления страниц
%\fancyhf{}                          % Очищаем текущие значения
%\fancyhead[C]{\thepage}             % Печатаем номер страницы на середине верхнего поля
%\renewcommand{\headrulewidth}{0pt}  % Убираем разделительную линию

%%% Оглавление %%%
%\renewcommand{\cftchapdotsep}{\cftdotsep}
%\renewcommand{\cftchapleader}{\cftdotfill{\cftdotsep}}
\renewcommand{\cftsecleader}{\cftdotfill{\cftdotsep}}
\renewcommand{\cftfigleader}{\cftdotfill{\cftdotsep}}
\renewcommand{\cfttableader}{\cftdotfill{\cftdotsep}}

\title{HarrixQtLibraryForLaTeX v.1.31}
\author{А.\,Б. Сергиенко}
\date{\today}


\begin{document}

%%% HarrixLaTeXDocumentTemplate
%%% Версия 1.10
%%% Шаблон документов в LaTeX на русском языке. Данный шаблон применяется в проектах HarrixTestFunctions, MathHarrixLibrary, Standard-Genetic-Algorithm  и др.
%%% https://github.com/Harrix/HarrixLaTeXDocumentTemplate
%%% Шаблон распространяется по лицензии Apache License, Version 2.0.

%%% Именования %%%
\renewcommand{\abstractname}{Аннотация}
\renewcommand{\alsoname}{см. также}
\renewcommand{\appendixname}{Приложение}
\renewcommand{\bibname}{Литература}
\renewcommand{\ccname}{исх.}
\renewcommand{\chaptername}{Глава}
%\renewcommand{\contentsname}{Содержание}
\renewcommand{\enclname}{вкл.}
\renewcommand{\figurename}{Рисунок}
\renewcommand{\headtoname}{вх.}
\renewcommand{\indexname}{Предметный указатель}
\renewcommand{\listfigurename}{Список рисунков}
\renewcommand{\listtablename}{Список таблиц}
\renewcommand{\pagename}{Стр.}
\renewcommand{\partname}{Часть}
\renewcommand{\refname}{Список литературы}
\renewcommand{\seename}{см.}
\renewcommand{\tablename}{Таблица}

%%% Псевдокоды %%%
% Перевод данных об алгоритмах
\renewcommand{\listalgorithmname}{Список алгоритмов}
\floatname{algorithm}{Алгоритм}

% Перевод команд псевдокода
\algrenewcommand\algorithmicwhile{\textbf{До тех пока}}
\algrenewcommand\algorithmicdo{\textbf{выполнять}}
\algrenewcommand\algorithmicrepeat{\textbf{Повторять}}
\algrenewcommand\algorithmicuntil{\textbf{Пока выполняется}}
\algrenewcommand\algorithmicend{\textbf{Конец}}
\algrenewcommand\algorithmicif{\textbf{Если}}
\algrenewcommand\algorithmicelse{\textbf{иначе}}
\algrenewcommand\algorithmicthen{\textbf{тогда}}
\algrenewcommand\algorithmicfor{\textbf{Цикл. }}
\algrenewcommand\algorithmicforall{\textbf{Выполнить для всех}}
\algrenewcommand\algorithmicfunction{\textbf{Функция}}
\algrenewcommand\algorithmicprocedure{\textbf{Процедура}}
\algrenewcommand\algorithmicloop{\textbf{Зациклить}}
\algrenewcommand\algorithmicrequire{\textbf{Условия:}}
\algrenewcommand\algorithmicensure{\textbf{Обеспечивающие условия:}}
\algrenewcommand\algorithmicreturn{\textbf{Возвратить}}
\algrenewtext{EndWhile}{\textbf{Конец цикла}}
\algrenewtext{EndLoop}{\textbf{Конец зацикливания}}
\algrenewtext{EndFor}{\textbf{Конец цикла}}
\algrenewtext{EndFunction}{\textbf{Конец функции}}
\algrenewtext{EndProcedure}{\textbf{Конец процедуры}}
\algrenewtext{EndIf}{\textbf{Конец условия}}
\algrenewtext{EndFor}{\textbf{Конец цикла}}
\algrenewtext{BeginAlgorithm}{\textbf{Начало алгоритма}}
\algrenewtext{EndAlgorithm}{\textbf{Конец алгоритма}}
\algrenewtext{BeginBlock}{\textbf{Начало блока. }}
\algrenewtext{EndBlock}{\textbf{Конец блока}}
\algrenewtext{ElsIf}{\textbf{иначе если }}

\maketitle

\begin{abstract}
Библиотека HarrixQtLibraryForLaTeX --- библиотека для отображения различных данных в LaTeX файлах.
\end{abstract}

\tableofcontents

\newpage

\section{Введение}

Библиотека HarrixQtLibraryForLaTeX --- это библиотека для отображения различных данных в LaTeX файлах.

Последнюю версию документа можно найти по адресу:

\href{https://github.com/Harrix/HarrixQtLibraryForLaTeX}{https://github.com/Harrix/HarrixQtLibraryForLaTeX}

Об установке библиотеки можно прочитать тут:

\href{http://blog.harrix.org/?p=1164}{http://blog.harrix.org/?p=1164}

С автором можно связаться по адресу \href{mailto:sergienkoanton@mail.ru}{sergienkoanton@mail.ru} или  \href{http://vk.com/harrix}{http://vk.com/harrix}.

Сайт автора, где публикуются последние новости: \href{http://blog.harrix.org/}{http://blog.harrix.org/}, а проекты располагаются по адресу \href{http://harrix.org/}{http://harrix.org/}.

%%%%%%%%%%%%%%%%%%%%%%%%%%%%%%%%%%%%%%%%%%%%%%%%%%%%%%%%%% ВСТАВЛЯТЬ НИЖЕ
\newpage
\section{Список функций}\label{section_listfunctions}
\textbf{Главные загрузочные функции}
\begin{enumerate}

\item \textbf{\hyperref[HQt_LatexBegin]{HQt\_LatexBegin}} --- Возвращает начало для полноценного Latex файла для шаблона https://github.com/Harrix/HarrixLaTeXDocumentTemplate

\item \textbf{\hyperref[HQt_LatexBeginArticle]{HQt\_LatexBeginArticle}} --- Возвращает начало для полноценного Latex файла в виде статьи для шаблона https://github.com/Harrix/HarrixLaTeXDocumentTemplate.

\item \textbf{\hyperref[HQt_LatexBeginArticleWithPgfplots]{HQt\_LatexBeginArticleWithPgfplots}} --- Возвращает начало для полноценного Latex файла в виде статьи для шаблона https://github.com/Harrix/HarrixLaTeXDocumentTemplate с использованием графиков через пакет pgfplots.

\item \textbf{\hyperref[HQt_LatexBeginWithPgfplots]{HQt\_LatexBeginWithPgfplots}} --- Возвращает начало для полноценного Latex файла для шаблона https://github.com/Harrix/HarrixLaTeXDocumentTemplate с использованием графиков через пакет pgfplots.

\item \textbf{\hyperref[HQt_LatexEnd]{HQt\_LatexEnd}} --- Возвращает концовку для полноценного Latex файла для шаблона https://github.com/Harrix/HarrixLaTeXDocumentTemplate

\end{enumerate}

\textbf{Графики}
\begin{enumerate}

\item \textbf{\hyperref[HQt_LatexDrawLine]{HQt\_LatexDrawLine}} --- Функция возвращает строку с Latex кодом отрисовки линии по функции Function.

\item \textbf{\hyperref[THQt_LatexDraw3DPlot]{THQt\_LatexDraw3DPlot}} --- Функция возвращает строку с Latex кодом отрисовки 3D поверхности по функции Function.

\item \textbf{\hyperref[THQt_LatexShow3DPlot]{THQt\_LatexShow3DPlot}} --- Функция возвращает строку с выводом некоторого 3D графика в виде поверхности.

\item \textbf{\hyperref[THQt_LatexShow3DPlotPoints]{THQt\_LatexShow3DPlotPoints}} --- Функция возвращает строку с выводом некоторого 3D графика в виде множества точек.

\item \textbf{\hyperref[THQt_LatexShowBar]{THQt\_LatexShowBar}} --- Функция возвращает строку с выводом некоторого графика гистограммы с Latex кодами.

\item \textbf{\hyperref[THQt_LatexShowChartOfLine]{THQt\_LatexShowChartOfLine}} --- Функция возвращает строку с выводом некоторого графика по точкам с Latex кодами.

\item \textbf{\hyperref[THQt_LatexShowChartsOfLineFromMatrix]{THQt\_LatexShowChartsOfLineFromMatrix}} --- Функция возвращает строку с выводом графиков из матрицы по точкам с Latex кодами.

\item \textbf{\hyperref[THQt_LatexShowIndependentChartsOfLineFromMatrix]{THQt\_LatexShowIndependentChartsOfLineFromMatrix}} --- Функция возвращает строку с выводом графиков из матрицы по точкам с Latex кодами. Нечетные столбцы --- это значения координат X графиков. Следующие за ними четные столбцы --- соответствующие значения Y. То есть графики друг от друга независимы.

\item \textbf{\hyperref[THQt_LatexShowTwoChartsOfLine]{THQt\_LatexShowTwoChartsOfLine}} --- Функция возвращает строку с выводом некоторых двух графиков по точкам с Latex кодами.

\item \textbf{\hyperref[THQt_LatexShowTwoIndependentChartsOfLine]{THQt\_LatexShowTwoIndependentChartsOfLine}} --- Функция возвращает строку с выводом некоторых двух независимых графиков по точкам с Latex кодами.

\item \textbf{\hyperref[THQt_LatexShowTwoIndependentChartsOfPointsAndLine]{THQt\_LatexShowTwoIndependentChartsOfPointsAndLine}} --- Функция возвращает строку с выводом некоторого двух независимых графиков по точкам с Latex кодами. Один график выводится в виде точек, а второй в виде линии. Удобно для отображения регрессий.

\end{enumerate}

\textbf{Обработка текста}
\begin{enumerate}

\item \textbf{\hyperref[HQt_ForcedWordWrap]{HQt\_ForcedWordWrap}} --- Функция расставляет принудительные переносы в стиле Latex.

\item \textbf{\hyperref[HQt_LatexGreenText]{HQt\_LatexGreenText}} --- Функция возвращает строку с выводом зеленого текста.

\item \textbf{\hyperref[HQt_LatexRedText]{HQt\_LatexRedText}} --- Функция возвращает строку с выводом красного текста.

\item \textbf{\hyperref[HQt_TextForLatexToText]{HQt\_TextForLatexToText}} --- Функция обрабатывает строку String из переделки функции HQt\_TextToTextForLatex в нормальную строку. Еще удаляются знаки \$, которые обрамляют формулы.

\item \textbf{\hyperref[HQt_TextToTextForLatex]{HQt\_TextToTextForLatex}} --- Функция переводит текст в текст, который можно добавить в Latex код. В-первую очередь, это экранирование некоторых элементов.

\item \textbf{\hyperref[THQt_LatexNumberToText]{THQt\_LatexNumberToText}} --- Функция выводит число VMHL\_X в строку Latex, причем число выделено жирным.

\end{enumerate}

\textbf{Показ математических выражений}
\begin{enumerate}

\item \textbf{\hyperref[THQt_LatexShowMatrix]{THQt\_LatexShowMatrix}} --- Функция возвращает строку с выводом некоторой матрицы VMHL\_Matrix с Latex кодами.

\item \textbf{\hyperref[THQt_LatexShowMatrix2]{THQt\_LatexShowMatrix2}} --- Функция возвращает строку с выводом некоторой матрицы VMHL\_Matrix с Latex кодами.

\item \textbf{\hyperref[THQt_LatexShowVector]{THQt\_LatexShowVector}} --- Функция возвращает строку с выводом некоторого вектора VMHL\_Vector с Latex кодами.

\item \textbf{\hyperref[THQt_LatexShowVector2]{THQt\_LatexShowVector2}} --- Функция возвращает строку с выводом некоторого вектора VMH\_Vector с Latex кодами.

\item \textbf{\hyperref[THQt_LatexShowVectorT]{THQt\_LatexShowVectorT}} --- Функция возвращает строку с выводом некоторого вектора VMHL\_Vector в транспонированном виде с Latex кодами.

\end{enumerate}

\textbf{Составные изображения}
\begin{enumerate}

\item \textbf{\hyperref[HQt_LatexBeginCompositionFigure]{HQt\_LatexBeginCompositionFigure}} --- Функция возвращает строку с выводом начала рисунка, состоящего из нескольких рисунков или графиков.

\item \textbf{\hyperref[HQt_LatexBeginFigureInCompositionFigure]{HQt\_LatexBeginFigureInCompositionFigure}} --- Функция возвращает строку с Latex кодом при добавлении дополнительного рисунка или графика в рисунок, состоящего из нескольких рисунков.

\item \textbf{\hyperref[HQt_LatexEndCompositionFigure]{HQt\_LatexEndCompositionFigure}} --- Функция возвращает строку с выводом окончания рисунка, состоящего из нескольких рисунков или графиков.

\item \textbf{\hyperref[HQt_LatexEndFigureInCompositionFigure]{HQt\_LatexEndFigureInCompositionFigure}} --- Функция возвращает строку с Latex кодом после добавлении дополнительного рисунка или графика в рисунок, состоящего из нескольких рисунков.

\end{enumerate}

\textbf{Таблицы}
\begin{enumerate}

\item \textbf{\hyperref[HQt_LatexShowTable]{HQt\_LatexShowTable}} --- Функция возвращает строку с выводом таблицы.

\end{enumerate}

\textbf{Текст}
\begin{enumerate}

\item \textbf{\hyperref[HQt_LatexShowAlert]{HQt\_LatexShowAlert}} --- Функция возвращает строку с выводом некоторого предупреждения.

\item \textbf{\hyperref[HQt_LatexShowHr]{HQt\_LatexShowHr}} --- Функция возвращает строку с выводом горизонтальной линии.

\item \textbf{\hyperref[HQt_LatexShowSection]{HQt\_LatexShowSection}} --- Функция возвращает строку с выводом некоторой строки в виде заголовка.

\item \textbf{\hyperref[HQt_LatexShowSimpleText]{HQt\_LatexShowSimpleText}} --- Функция возвращает строку с выводом некоторой строки с Latex кодами без всякого излишества.

\item \textbf{\hyperref[HQt_LatexShowSubsection]{HQt\_LatexShowSubsection}} --- Функция возвращает строку с выводом некоторой строки в виде подзаголовка.

\item \textbf{\hyperref[HQt_LatexShowText]{HQt\_LatexShowText}} --- Функция возвращает строку с выводом некоторой строки с Latex кодами.

\item \textbf{\hyperref[THQt_LatexShowNumber]{THQt\_LatexShowNumber}} --- Функция возвращает строку с выводом некоторого числа VMHL\_X с Latex кодами.

\end{enumerate}


\newpage
\section{Функции}
\subsection{Главные загрузочные функции}

\subsubsection{HQt\_LatexBegin}\label{HQt_LatexBegin}

Возвращает начало для полноценного Latex файла для шаблона https://github.com/Harrix/HarrixLaTeXDocumentTemplate


\begin{lstlisting}[label=code_syntax_HQt_LatexBegin,caption=Синтаксис]
QString HQt_LatexBegin();
\end{lstlisting}

\textbf{Входные параметры:}

Отсутствует.

\textbf{Возвращаемое значение:}

Начало для полноценного Latex файла.


\subsubsection{HQt\_LatexBeginArticle}\label{HQt_LatexBeginArticle}

Возвращает начало для полноценного Latex файла в виде статьи для шаблона https://github.com/Harrix/HarrixLaTeXDocumentTemplate.


\begin{lstlisting}[label=code_syntax_HQt_LatexBeginArticle,caption=Синтаксис]
QString HQt_LatexBeginArticle();
\end{lstlisting}

\textbf{Входные параметры:}

Отсутствует.

\textbf{Возвращаемое значение:}

Начало для полноценного Latex файла в виде статьи.


\subsubsection{HQt\_LatexBeginArticleWithPgfplots}\label{HQt_LatexBeginArticleWithPgfplots}

Возвращает начало для полноценного Latex файла в виде статьи для шаблона https://github.com/Harrix/HarrixLaTeXDocumentTemplate с использованием графиков через пакет pgfplots.


\begin{lstlisting}[label=code_syntax_HQt_LatexBeginArticleWithPgfplots,caption=Синтаксис]
QString HQt_LatexBeginArticleWithPgfplots();
\end{lstlisting}

\textbf{Входные параметры:}

Отсутствует.

\textbf{Возвращаемое значение:}

Начало для полноценного Latex файла в виде статьи с использованием графиков через пакет pgfplots.


\subsubsection{HQt\_LatexBeginWithPgfplots}\label{HQt_LatexBeginWithPgfplots}

Возвращает начало для полноценного Latex файла для шаблона https://github.com/Harrix/HarrixLaTeXDocumentTemplate с использованием графиков через пакет pgfplots.


\begin{lstlisting}[label=code_syntax_HQt_LatexBeginWithPgfplots,caption=Синтаксис]
QString HQt_LatexBeginWithPgfplots();
\end{lstlisting}

\textbf{Входные параметры:}

Отсутствует.

\textbf{Возвращаемое значение:}

Начало для полноценного Latex файла с использованием графиков через пакет pgfplots.


\subsubsection{HQt\_LatexEnd}\label{HQt_LatexEnd}

Возвращает концовку для полноценного Latex файла для шаблона https://github.com/Harrix/HarrixLaTeXDocumentTemplate


\begin{lstlisting}[label=code_syntax_HQt_LatexEnd,caption=Синтаксис]
QString HQt_LatexEnd();
\end{lstlisting}

\textbf{Входные параметры:}

Отсутствует.

\textbf{Возвращаемое значение:}

Концовка для полноценного Latex файла.


\subsection{Графики}

\subsubsection{HQt\_LatexDrawLine}\label{HQt_LatexDrawLine}

Функция возвращает строку с Latex кодом отрисовки линии по функции Function.


\begin{lstlisting}[label=code_syntax_HQt_LatexDrawLine,caption=Синтаксис]
QString HQt_LatexDrawLine (double Left, double Right, double h, double (*Function)(double), QString TitleChart, QString NameVectorX, QString NameVectorY, QString NameLine, bool ShowLine, bool ShowPoints, bool ShowArea, bool ShowSpecPoints, bool RedLine);
QString HQt_LatexDrawLine (double Left, double Right, double h, double (*Function)(double), QString TitleChart, QString NameVectorX, QString NameVectorY, bool ShowLine, bool ShowPoints, bool ShowArea, bool ShowSpecPoints, bool RedLine);
QString HQt_LatexDrawLine (double Left, double Right, double h, double (*Function)(double), QString TitleChart, QString NameVectorX, QString NameVectorY, QString NameLine);
QString HQt_LatexDrawLine (double Left, double Right, double h, double (*Function)(double));
\end{lstlisting}

\textbf{Входные параметры:}
 
    Left --- левая граница области;
 
    Right --- правая граница области;
 
    h --- шаг, с которым надо рисовать график;
 
    Function --- указатель на вычисляемую функцию;
 
    TitleChart --- заголовок графика;
 
    NameVectorX --- название оси Ox. В формате: [обозначение], [расшифровка]. Например: u, Вероятность выбора;
 
    NameVectorY --- название оси Oy. В формате: [обозначение], [расшифровка]. Например: q, Количество абрикосов;
 
    NameLine --- название первого графика (для легенды);
 
    ShowLine --- показывать ли линию;
 
    ShowPoints --- показывать ли точки;
 
    ShowArea --- показывать ли закрашенную область под кривой;
 
    ShowSpecPoints --- показывать ли специальные точки;
 
    RedLine --- рисовать ли красную линию, или синюю.

\textbf{Возвращаемое значение:}

Строка с Latex кодом.


\subsubsection{THQt\_LatexDraw3DPlot}\label{THQt_LatexDraw3DPlot}

Функция возвращает строку с Latex кодом отрисовки 3D поверхности по функции Function.


\begin{lstlisting}[label=code_syntax_THQt_LatexDraw3DPlot,caption=Синтаксис]
QString THQt_LatexDraw3DPlot (double Left_X, double Right_X, double Left_Y, double Right_Y, int N, double (*Function)(double, double),  QString TitleChart, QString NameVectorX, QString NameVectorY, QString NameVectorZ, QString Label, QString ColorMap, TypeOf3DPlot Type, double Opacity, double AngleHorizontal, double AngleVertical, bool ColorBar, bool ForNormalSize);
QString THQt_LatexDraw3DPlot (double Left_X, double Right_X, double Left_Y, double Right_Y, int N, double (*Function)(double, double),  QString TitleChart, QString NameVectorX, QString NameVectorY, QString NameVectorZ, QString Label, QString ColorMap, TypeOf3DPlot Type, bool ColorBar, bool ForNormalSize);
QString THQt_LatexDraw3DPlot (double Left_X, double Right_X, double Left_Y, double Right_Y, int N, double (*Function)(double, double),  QString TitleChart, QString NameVectorX, QString NameVectorY, QString NameVectorZ, QString Label, QString ColorMap, TypeOf3DPlot Type, bool ColorBar);
QString THQt_LatexDraw3DPlot (double Left_X, double Right_X, double Left_Y, double Right_Y, int N, double (*Function)(double, double),  QString TitleChart, QString NameVectorX, QString NameVectorY, QString NameVectorZ, QString Label, QString ColorMap, TypeOf3DPlot Type);
QString THQt_LatexDraw3DPlot (double Left_X, double Right_X, double Left_Y, double Right_Y, int N, double (*Function)(double, double),  QString TitleChart, QString NameVectorX, QString NameVectorY, QString NameVectorZ, QString Label);
QString THQt_LatexDraw3DPlot (double Left_X, double Right_X, double Left_Y, double Right_Y, int N, double (*Function)(double, double));
QString THQt_LatexDraw3DPlot (double Left, double Right, int N, double (*Function)(double, double));
\end{lstlisting}

\textbf{Входные параметры:}

Left\_X --- левая граница по оси Ox;
 
    Right\_X --- правая граница по оси Ox;
 
    Left\_Y --- левая граница по оси Oy;
 
    Right\_Y --- правая граница по оси Oy;
 
    N --- сколько нужно построить точек по каждой оси. В итоге получим N*N точек;
 
    Function --- ссылка на отрисовываемую двумерную функцию;
 
    TitleChart --- заголовок графика;
 
    NameVectorX --- название оси Ox. В формате: [обозначение], [расшифровка]. Например: u, Вероятность выбора;
 
    NameVectorY --- название оси Oy. В формате: [обозначение], [расшифровка]. Например: q, Количество абрикосов;
 
    NameVectorZ --- название оси Oz. В формате: [обозначение], [расшифровка]. Например: z, Вероятность;
 
    Label --- label для графика;
 
    ColorMap --- какой раскраски будет график. Возможны значения: mathcad, matlab, hot или тот, что вы хотите использовать. Рекомендуется mathcad;
 
    Type --- тип графика. Возможные значения:
 
       Plot3D\_Points --- в виде точек,
 
       Plot3D\_Surface --- в виде поверхности с непрерывной заливкой,
 
       Plot3D\_SurfaceGrid --- в виде поверхности с сеточной заливкой,
 
       Plot3D\_TopView --- вид сверху;
 
    Opacity --- прозрачность графика. Может изменяться от 0 до 1;
 
    AngleHorizontal --- угол поворота графика по горизонтали в градусах от ---180 до 180. Рекомендуется 25;
 
    AngleVertical --- угол поворота графика по вертикали в градусах от ---180 до 180. Рекомендуется 30;
 
    ColorBar --- рисовать с графиком колонку с градациями цветов или нет;
 
    ForNormalSize --- нормальный размер графика (на всю ширину), или для маленького размера график создается.

\textbf{Возвращаемое значение:}

Строка с Latex кодом.


\subsubsection{THQt\_LatexShow3DPlot}\label{THQt_LatexShow3DPlot}

Функция возвращает строку с выводом некоторого 3D графика в виде поверхности.


\begin{lstlisting}[label=code_syntax_THQt_LatexShow3DPlot,caption=Синтаксис]
template <class T> QString THQt_LatexShow3DPlot (T *VMHL_VectorX, T *VMHL_VectorY, T **VMHL_VectorZ,  int VMHL_N,  int VMHL_M, QString TitleChart, QString NameVectorX, QString NameVectorY, QString NameVectorZ, QString Label, QString ColorMap, TypeOf3DPlot Type, double Opacity, double AngleHorizontal, double AngleVertical, bool ColorBar, bool ForNormalSize);
template <class T> QString THQt_LatexShow3DPlot (T *VMHL_VectorX, T *VMHL_VectorY, T **VMHL_VectorZ,  int VMHL_N,  int VMHL_M, QString TitleChart, QString NameVectorX, QString NameVectorY, QString NameVectorZ, QString Label, QString ColorMap, TypeOf3DPlot Type, bool ColorBar, bool ForNormalSize);
template <class T> QString THQt_LatexShow3DPlot (T *VMHL_VectorX, T *VMHL_VectorY, T **VMHL_VectorZ,  int VMHL_N,  int VMHL_M, QString TitleChart, QString NameVectorX, QString NameVectorY, QString NameVectorZ, QString Label, QString ColorMap, TypeOf3DPlot Type, bool ColorBar);
template <class T> QString THQt_LatexShow3DPlot (T *VMHL_VectorX, T *VMHL_VectorY, T **VMHL_VectorZ,  int VMHL_N,  int VMHL_M, QString TitleChart, QString NameVectorX, QString NameVectorY, QString NameVectorZ, QString Label, QString ColorMap, TypeOf3DPlot Type);
template <class T> QString THQt_LatexShow3DPlot (T *VMHL_VectorX, T *VMHL_VectorY, T **VMHL_VectorZ,  int VMHL_N,  int VMHL_M, QString TitleChart, QString NameVectorX, QString NameVectorY, QString NameVectorZ, QString Label);
template <class T> QString THQt_LatexShow3DPlot (T *VMHL_VectorX, T *VMHL_VectorY, T **VMHL_VectorZ,  int VMHL_N,  int VMHL_M);
\end{lstlisting}

\textbf{Входные параметры:}
 

 
    VMHL\_VectorX --- указатель на вектор значений координат X сетки точек. Количество элементов VMHL\_N;
 
    VMHL\_VectorY --- указатель на вектор значений координат Y сетки точек. Количество элементов VMHL\_M;
 
    VMHL\_VectorZ --- указатель на матрицу значений координат Z точек. Количество элементов VMHL\_NxVMHL\_M;
 
    VMHL\_N --- количество значений в сетке по оси Ox;
 
    VMHL\_M --- количество значений в сетке по оси Oy;
 
    TitleChart --- заголовок графика;
 
    NameVectorX --- название оси Ox. В формате: [обозначение], [расшифровка]. Например: u, Вероятность выбора;
 
    NameVectorY --- название оси Oy. В формате: [обозначение], [расшифровка]. Например: q, Количество абрикосов;
 
    NameVectorZ --- название оси Oz. В формате: [обозначение], [расшифровка]. Например: z, Вероятность;
 
    Label --- label для графика;
 
    ColorMap --- какой раскраски будет график. Возможны значения: mathcad, matlab, hot или тот, что вы хотите использовать. Рекомендуется mathcad;
 
    Type --- тип графика. Возможные значения:
 
       Plot3D\_Points --- в виде точек,
 
       Plot3D\_Surface --- в виде поверхности с непрерывной заливкой,
 
       Plot3D\_SurfaceGrid --- в виде поверхности с сеточной заливкой,
 
       Plot3D\_TopView --- вид сверху;
 
    Opacity --- прозначность графика. Может изменяться от 0 до 1;
 
    AngleHorizontal --- угол поворота графика по горизонтали в градусах от ---180 до 180. Рекомендуется 25;
 
    AngleVertical --- угол поворота графика по вертикали в градусах от ---180 до 180. Рекомендуется 30;
 
    ColorBar --- рисоватm с графиком колонку с градациями цветов или нет;
 
    ForNormalSize --- нормальный размер графика (на всю ширину), или для маленького размера график создается.
	
\textbf{Возвращаемое значение:}

Строка с Latex кодами с выводимым графиком.


\subsubsection{THQt\_LatexShow3DPlotPoints}\label{THQt_LatexShow3DPlotPoints}

Функция возвращает строку с выводом некоторого 3D графика в виде множества точек.


\begin{lstlisting}[label=code_syntax_THQt_LatexShow3DPlotPoints,caption=Синтаксис]
template <class T> QString THQt_LatexShow3DPlotPoints (T *VMHL_VectorX, T *VMHL_VectorY, T *VMHL_VectorZ,  int VMHL_N, QString TitleChart, QString NameVectorX, QString NameVectorY, QString NameVectorZ, QString Label, QString ColorMap, bool ForNormalSize);
template <class T> QString THQt_LatexShow3DPlotPoints (T *VMHL_VectorX, T *VMHL_VectorY, T *VMHL_VectorZ,  int VMHL_N, QString TitleChart, QString NameVectorX, QString NameVectorY, QString NameVectorZ, QString Label, bool ForNormalSize);
template <class T> QString THQt_LatexShow3DPlotPoints (T *VMHL_VectorX, T *VMHL_VectorY, T *VMHL_VectorZ,  int VMHL_N, QString TitleChart, QString NameVectorX, QString NameVectorY, QString NameVectorZ, QString Label);
template <class T> QString THQt_LatexShow3DPlotPoints (T *VMHL_VectorX, T *VMHL_VectorY, T *VMHL_VectorZ,  int VMHL_N);
\end{lstlisting}

\textbf{Входные параметры:}
 

VMHL\_VectorX --- указатель на вектор координат X точек;
 
    VMHL\_VectorY --- указатель на вектор координат Y точек;
 
    VMHL\_VectorZ --- указатель на вектор координат Z точек;
 
    VMHL\_N --- количество точек;
 
    TitleChart --- заголовок графика;
 
    NameVectorX --- название оси Ox. В формате: [обозначение], [расшифровка]. Например: u, Вероятность выбора;
 
    NameVectorY --- название оси Oy. В формате: [обозначение], [расшифровка]. Например: q, Количество абрикосов;
 
    NameVectorZ --- название оси Oz. В формате: [обозначение], [расшифровка]. Например: z, Вероятность;
 
    Label --- label для графика;
 
    ColorMap --- какой раскраски будет график. Возможны значения: mathcad, matlab, hot или тот, что вы хотите использовать. Рекомендуется mathcad.
 
    ForNormalSize --- нормальный размер графика (на всю ширину), или для маленького размера график создается.
	
\textbf{Возвращаемое значение:}

Строка с Latex кодами с выводимым графиком.


\subsubsection{THQt\_LatexShowBar}\label{THQt_LatexShowBar}

Функция возвращает строку с выводом некоторого графика гистограммы с Latex кодами.


\begin{lstlisting}[label=code_syntax_THQt_LatexShowBar,caption=Синтаксис]
template <class T> QString THQt_LatexShowBar (T *VMHL_Vector, int VMHL_N, QString TitleChart, QString *NameVectorX, QString NameVectorY, QString Label, bool ForNormalSize, bool MinZero);
template <class T> QString THQt_LatexShowBar (T *VMHL_Vector, int VMHL_N, QString TitleChart, QString *NameVectorX, QString NameVectorY, QString Label, bool ForNormalSize);
template <class T> QString THQt_LatexShowBar (T *VMHL_Vector, int VMHL_N, QString TitleChart, QString *NameVectorX, QString NameVectorY, QString Label);
template <class T> QString THQt_LatexShowBar (T *VMHL_Vector, int VMHL_N);
template <class T> QString THQt_LatexShowBar (T *VMHL_Vector, int VMHL_N, QString TitleChart, QStringList NameVectorX, QString NameVectorY, QString Label, bool ForNormalSize, bool MinZero);
template <class T> QString THQt_LatexShowBar (T *VMHL_Vector, int VMHL_N, QString TitleChart, QStringList NameVectorX, QString NameVectorY, QString Label, bool ForNormalSize);
template <class T> QString THQt_LatexShowBar (T *VMHL_Vector, int VMHL_N, QString TitleChart, QStringList NameVectorX, QString NameVectorY, QString Label);
\end{lstlisting}

\textbf{Входные параметры:}
 
    VMHL\_Vector --- указатель на вектор значений точек;
 
    VMHL\_N --- количество точек;
 
    TitleChart --- заголовок графика;
 
    NameVectorX --- название значений точек. Будут подписаны под каждым столбиком на оси Ox. Количество элементов VMHL\_N;
 
    NameVectorY --- название оси Oy. В формате: [обозначение], [расшифровка]. Например: q, Количество абрикосов;
 
    Label --- label для графика;
 
    ForNormalSize --- нормальный размер графика (на всю ширину), или для маленького размера график создается;
 
    MinZero --- гистограмму начинать с нуля (true) или с минимального значения среди VMHL\_Vector (false).
	
\textbf{Возвращаемое значение:}

Строка с Latex кодами с выводимым графиком.


\subsubsection{THQt\_LatexShowChartOfLine}\label{THQt_LatexShowChartOfLine}

Функция возвращает строку с выводом некоторого графика по точкам с Latex кодами.


\begin{lstlisting}[label=code_syntax_THQt_LatexShowChartOfLine,caption=Синтаксис]
template <class T> QString THQt_LatexShowChartOfLine (T *VMHL_VectorX,T *VMHL_VectorY, int VMHL_N, QString TitleChart, QString NameVectorX, QString NameVectorY, QString NameLine, QString Label, bool ShowLine, bool ShowPoints, bool ShowArea, bool ShowSpecPoints, bool RedLine, bool ForNormalSize);
template <class T> QString THQt_LatexShowChartOfLine (T *VMHL_VectorX,T *VMHL_VectorY, int VMHL_N, QString TitleChart, QString NameVectorX, QString NameVectorY, QString NameLine, QString Label, bool ShowLine, bool ShowPoints, bool ShowArea, bool ShowSpecPoints, bool RedLine);
template <class T> QString THQt_LatexShowChartOfLine (T *VMHL_VectorX,T *VMHL_VectorY, int VMHL_N, QString TitleChart, QString NameVectorX, QString NameVectorY, QString NameLine, QString Label, bool ShowLine, bool ShowPoints, bool ShowArea, bool ShowSpecPoints);
template <class T> QString THQt_LatexShowChartOfLine (T *VMHL_VectorX,T *VMHL_VectorY, int VMHL_N, QString TitleChart, QString NameVectorX, QString NameVectorY, QString NameLine, QString Label);
template <class T> QString THQt_LatexShowChartOfLine (T *VMHL_VectorX,T *VMHL_VectorY, int VMHL_N);
\end{lstlisting}

\textbf{Входные параметры:}

    VMHL\_VectorX --- указатель на вектор координат X точек;
 
    VMHL\_VectorY --- указатель на вектор координат Y точек;
 
    VMHL\_N --- количество точек;
 
    TitleChart --- заголовок графика;
 
    NameVectorX --- название оси Ox. В формате: [обозначение], [расшифровка]. Например: u, Вероятность выбора;
 
    NameVectorY --- название оси Oy. В формате: [обозначение], [расшифровка]. Например: q, Количество абрикосов;
 
    NameLine --- название первого графика (для легенды);
 
    Label --- label для графика
 
    ShowLine --- показывать ли линию;
 
    ShowPoints --- показывать ли точки;
 
    ShowArea --- показывать ли закрашенную область под кривой;
 
    ShowSpecPoints --- показывать ли специальные точки;
 
    RedLine --- рисовать ли красную линию, или синюю;
 
    ForNormalSize --- нормальный размер графика (на всю ширину), или для маленького размера график создается.
	
\textbf{Возвращаемое значение:}

Строка с Latex кодами с выводимым графиком.


\subsubsection{THQt\_LatexShowChartsOfLineFromMatrix}\label{THQt_LatexShowChartsOfLineFromMatrix}

Функция возвращает строку с выводом графиков из матрицы по точкам с Latex кодами.


\begin{lstlisting}[label=code_syntax_THQt_LatexShowChartsOfLineFromMatrix,caption=Синтаксис]
template <class T> QString THQt_LatexShowChartsOfLineFromMatrix (T **VMHL_MatrixXY,int VMHL_N,int VMHL_M, QString TitleChart, QString NameVectorX, QString NameVectorY,QString *NameLine, QString Label, bool ShowLine,bool ShowPoints,bool ShowArea,bool ShowSpecPoints, bool ForNormalSize, bool GrayStyle, bool SolidStyle, bool CircleStyle);
template <class T> QString THQt_LatexShowChartsOfLineFromMatrix (T **VMHL_MatrixXY,int VMHL_N,int VMHL_M, QString TitleChart, QString NameVectorX, QString NameVectorY,QString *NameLine, QString Label, bool ShowLine,bool ShowPoints,bool ShowArea,bool ShowSpecPoints, bool ForNormalSize);
template <class T> QString THQt_LatexShowChartsOfLineFromMatrix (T **VMHL_MatrixXY,int VMHL_N,int VMHL_M, QString TitleChart, QString NameVectorX, QString NameVectorY,QString *NameLine, QString Label, bool ShowLine,bool ShowPoints,bool ShowArea,bool ShowSpecPoints);
template <class T> QString THQt_LatexShowChartsOfLineFromMatrix (T **VMHL_MatrixXY,int VMHL_N,int VMHL_M, QString TitleChart, QString NameVectorX, QString NameVectorY,QString *NameLine, QString Label);
template <class T> QString THQt_LatexShowChartsOfLineFromMatrix (T **VMHL_MatrixXY,int VMHL_N,int VMHL_M);
\end{lstlisting}

\textbf{Входные параметры:}
 
VMHL\_MatrixXY --- указатель на матрицу значений X и Y графиков;
 
    VMHL\_N --- количество точек;
 
    VMHL\_M --- количество столбцов матрицы (1+количество графиков);
 
    TitleChart --- заголовок графика;
 
    NameVectorX --- название оси Ox. В формате: [обозначение], [расшифровка]. Например: u, Вероятность выбора;;
 
    NameVectorY --- название оси Oy. В формате: [обозначение], [расшифровка]. Например: q, Количество абрикосов;
 
    NameLine --- указатель на вектор названий графиков (для легенды) количество элементов VMHL\_M---1 (так как первый столбец --- это X значения);
 
    Label --- label для графика;
 
    ShowLine --- показывать ли линию;
 
    ShowPoints --- показывать ли точки;
 
    ShowArea --- показывать ли закрашенную область под кривой;
 
    ShowSpecPoints --- показывать ли специальные точки;
 
    ForNormalSize --- нормальный размер графика (на всю ширину) или для маленького размера график создается;
 
    GrayStyle --- серый стиль графиков;
 
    SolidStyle --- линии делать сплошными или разными по типу (точками, тире и др.);
 
    CircleStyle --- точки все делать кругляшками или нет.
	
\textbf{Возвращаемое значение:}

Строка с Latex кодами с выводимым графиком.


\subsubsection{THQt\_LatexShowIndependentChartsOfLineFromMatrix}\label{THQt_LatexShowIndependentChartsOfLineFromMatrix}

Функция возвращает строку с выводом графиков из матрицы по точкам с Latex кодами. Нечетные столбцы --- это значения координат X графиков. Следующие за ними четные столбцы --- соответствующие значения Y. То есть графики друг от друга независимы.


\begin{lstlisting}[label=code_syntax_THQt_LatexShowIndependentChartsOfLineFromMatrix,caption=Синтаксис]
template <class T> QString THQt_LatexShowIndependentChartsOfLineFromMatrix (T **VMHL_MatrixXY,int *VMHL_N_EveryCol,int VMHL_M, QString TitleChart, QString NameVectorX, QString NameVectorY,QString *NameLine, QString Label, bool ShowLine,bool ShowPoints,bool ShowArea,bool ShowSpecPoints, bool ForNormalSize, bool GrayStyle, bool SolidStyle, bool CircleStyle);
template <class T> QString THQt_LatexShowIndependentChartsOfLineFromMatrix (T **VMHL_MatrixXY,int *VMHL_N_EveryCol,int VMHL_M, QString TitleChart, QString NameVectorX, QString NameVectorY,QString *NameLine, QString Label, bool ShowLine,bool ShowPoints,bool ShowArea,bool ShowSpecPoints, bool ForNormalSize);
template <class T> QString THQt_LatexShowIndependentChartsOfLineFromMatrix (T **VMHL_MatrixXY,int *VMHL_N_EveryCol,int VMHL_M, QString TitleChart, QString NameVectorX, QString NameVectorY,QString *NameLine, QString Label, bool ShowLine,bool ShowPoints,bool ShowArea,bool ShowSpecPoints);
template <class T> QString THQt_LatexShowIndependentChartsOfLineFromMatrix (T **VMHL_MatrixXY,int *VMHL_N_EveryCol,int VMHL_M, QString TitleChart, QString NameVectorX, QString NameVectorY,QString *NameLine, QString Label);
template <class T> QString THQt_LatexShowIndependentChartsOfLineFromMatrix (T **VMHL_MatrixXY,int *VMHL_N_EveryCol,int VMHL_M);
template <class T> QString THQt_LatexShowIndependentChartsOfLineFromMatrix (T **VMHL_MatrixXY,int VMHL_N,int VMHL_M, QString TitleChart, QString NameVectorX, QString NameVectorY,QString *NameLine, QString Label, bool ShowLine,bool ShowPoints,bool ShowArea,bool ShowSpecPoints, bool ForNormalSize, bool GrayStyle, bool SolidStyle, bool CircleStyle);
template <class T> QString THQt_LatexShowIndependentChartsOfLineFromMatrix (T **VMHL_MatrixXY,int VMHL_N,int VMHL_M, QString TitleChart, QString NameVectorX, QString NameVectorY,QString *NameLine, QString Label, bool ShowLine,bool ShowPoints,bool ShowArea,bool ShowSpecPoints, bool ForNormalSize);
template <class T> QString THQt_LatexShowIndependentChartsOfLineFromMatrix (T **VMHL_MatrixXY,int VMHL_N,int VMHL_M, QString TitleChart, QString NameVectorX, QString NameVectorY,QString *NameLine, QString Label, bool ShowLine,bool ShowPoints,bool ShowArea,bool ShowSpecPoints);
template <class T> QString THQt_LatexShowIndependentChartsOfLineFromMatrix (T **VMHL_MatrixXY,int VMHL_N,int VMHL_M, QString TitleChart, QString NameVectorX, QString NameVectorY,QString *NameLine, QString Label);
template <class T> QString THQt_LatexShowIndependentChartsOfLineFromMatrix (T **VMHL_MatrixXY,int VMHL_N,int VMHL_M);
\end{lstlisting}

\textbf{Входные параметры:}
 
VMHL\_MatrixXY --- указатель на матрицу значений X и Y графиков;
 
VMHL\_N\_EveryCol --- количество элементов в каждом столбце (так как столбцы идут по парам, то число элементов в нечетном и
 
следующем за ним четном столбце должны совпадать, например 10,10,5,5,7,7);
 
VMHL\_M --- количество столбцов матрицы (должно быть четным числом конечно);
 
TitleChart --- заголовок графика;
 
NameVectorX --- название оси Ox. В формате: [обозначение], [расшифровка]. Например: u, Вероятность выбора;
 
NameVectorY --- название оси Oy. В формате: [обозначение], [расшифровка]. Например: q, Количество абрикосов;
 
NameLine --- указатель на вектор названий графиков (для легенды) количество элементов VMHL\_M/2;
 
Label --- label для графика;
 
ShowLine --- показывать ли линию;
 
ShowPoints --- показывать ли точки;
 
ShowArea --- показывать ли закрашенную область под кривой;
 
ShowSpecPoints --- показывать ли специальные точки;
 
ForNormalSize --- нормальный размер графика (на всю ширину) или для маленького размера график создается;
 
GrayStyle --- серый стиль графиков;
 
SolidStyle --- линии делать сплошными или разными по типу (точками, тире и др.);
 
CircleStyle --- точки все делать кругляшками или нет.
	
\textbf{Возвращаемое значение:}

Строка с Latex кодами с выводимым графиком.


\subsubsection{THQt\_LatexShowTwoChartsOfLine}\label{THQt_LatexShowTwoChartsOfLine}

Функция возвращает строку с выводом некоторых двух графиков по точкам с Latex кодами.


\begin{lstlisting}[label=code_syntax_THQt_LatexShowTwoChartsOfLine,caption=Синтаксис]
template <class T> QString THQt_LatexShowTwoChartsOfLine (T *VMHL_VectorX,T *VMHL_VectorY1,T *VMHL_VectorY2, int VMHL_N, QString TitleChart, QString NameVectorX, QString NameVectorY,QString NameLine1, QString NameLine2, QString Label,bool ShowLine,bool ShowPoints,bool ShowArea,bool ShowSpecPoints, bool ForNormalSize, bool GrayStyle);
template <class T> QString THQt_LatexShowTwoChartsOfLine (T *VMHL_VectorX,T *VMHL_VectorY1,T *VMHL_VectorY2, int VMHL_N, QString TitleChart, QString NameVectorX, QString NameVectorY,QString NameLine1, QString NameLine2, QString Label,bool ShowLine,bool ShowPoints,bool ShowArea,bool ShowSpecPoints, bool ForNormalSize);
template <class T> QString THQt_LatexShowTwoChartsOfLine (T *VMHL_VectorX,T *VMHL_VectorY1,T *VMHL_VectorY2, int VMHL_N, QString TitleChart, QString NameVectorX, QString NameVectorY,QString NameLine1, QString NameLine2, QString Label,bool ShowLine,bool ShowPoints,bool ShowArea,bool ShowSpecPoints);
template <class T> QString THQt_LatexShowTwoChartsOfLine (T *VMHL_VectorX,T *VMHL_VectorY1,T *VMHL_VectorY2, int VMHL_N, QString TitleChart, QString NameVectorX, QString NameVectorY,QString NameLine1, QString NameLine2, QString Label);
template <class T> QString THQt_LatexShowTwoChartsOfLine (T *VMHL_VectorX,T *VMHL_VectorY1,T *VMHL_VectorY2, int VMHL_N);
\end{lstlisting}

\textbf{Входные параметры:}
 
    VMHL\_VectorX --- указатель на вектор координат X точек;
 
    VMHL\_VectorY1 --- указатель на вектор координат Y точек первой линии;
 
    VMHL\_VectorY2 --- указатель на вектор координат Y точек второй линии;
 
    VMHL\_N --- количество точек;
 
    TitleChart --- заголовок графика;
 
    NameVectorX --- название оси Ox. В формате: [обозначение], [расшифровка]. Например: u, Вероятность выбора;
 
    NameVectorY --- название оси Oy. В формате: [обозначение], [расшифровка]. Например: q, Количество абрикосов;
 
    NameLine1 --- название первого графика (для легенды);
 
    NameLine2 --- название второго графика (для легенды);
 
    Label --- label для графика;
 
    ShowLine --- показывать ли линию;
 
    ShowPoints --- показывать ли точки;
 
    ShowArea --- показывать ли закрашенную область под кривой;
 
    ShowSpecPoints --- показывать ли специальные точки;
 
    ForNormalSize --- нормальный размер графика (на всю ширину) или для маленького размера график создается;
 
    GrayStyle --- второй график рисовать серым, а не красным.
	
\textbf{Возвращаемое значение:}

Строка с Latex кодами с выводимым графиком.


\subsubsection{THQt\_LatexShowTwoIndependentChartsOfLine}\label{THQt_LatexShowTwoIndependentChartsOfLine}

Функция возвращает строку с выводом некоторых двух независимых графиков по точкам с Latex кодами.


\begin{lstlisting}[label=code_syntax_THQt_LatexShowTwoIndependentChartsOfLine,caption=Синтаксис]
template <class T> QString THQt_LatexShowTwoIndependentChartsOfLine (T *VMHL_VectorX1,T *VMHL_VectorY1,int VMHL_N1,T *VMHL_VectorX2,T *VMHL_VectorY2, int VMHL_N2, QString TitleChart, QString NameVectorX, QString NameVectorY,QString NameLine1, QString NameLine2, QString Label, bool ShowLine,bool ShowPoints,bool ShowArea,bool ShowSpecPoints, bool ForNormalSize, bool GrayStyle);
template <class T> QString THQt_LatexShowTwoIndependentChartsOfLine (T *VMHL_VectorX1,T *VMHL_VectorY1,int VMHL_N1,T *VMHL_VectorX2,T *VMHL_VectorY2, int VMHL_N2, QString TitleChart, QString NameVectorX, QString NameVectorY,QString NameLine1, QString NameLine2, QString Label, bool ShowLine,bool ShowPoints,bool ShowArea,bool ShowSpecPoints, bool ForNormalSize);
template <class T> QString THQt_LatexShowTwoIndependentChartsOfLine (T *VMHL_VectorX1,T *VMHL_VectorY1,int VMHL_N1,T *VMHL_VectorX2,T *VMHL_VectorY2, int VMHL_N2, QString TitleChart, QString NameVectorX, QString NameVectorY,QString NameLine1, QString NameLine2, QString Label, bool ShowLine,bool ShowPoints,bool ShowArea,bool ShowSpecPoints);
template <class T> QString THQt_LatexShowTwoIndependentChartsOfLine (T *VMHL_VectorX1,T *VMHL_VectorY1,int VMHL_N1,T *VMHL_VectorX2,T *VMHL_VectorY2, int VMHL_N2, QString TitleChart, QString NameVectorX, QString NameVectorY,QString NameLine1, QString NameLine2, QString Label);
template <class T> QString THQt_LatexShowTwoIndependentChartsOfLine (T *VMHL_VectorX1,T *VMHL_VectorY1,int VMHL_N1,T *VMHL_VectorX2,T *VMHL_VectorY2, int VMHL_N2);
\end{lstlisting}

\textbf{Входные параметры:}
 
    VMHL\_VectorX1 --- указатель на вектор координат X точек первой линии;
 
    VMHL\_VectorY1 --- указатель на вектор координат Y точек первой линии;
 
    VMHL\_N1 --- количество точек первой линии;
 
    VMHL\_VectorX2 --- указатель на вектор координат X точек второй линии;
 
    VMHL\_VectorY2 --- указатель на вектор координат Y точек второй линии;
 
    VMHL\_N2 --- количество точек второй линии;
 
    TitleChart --- заголовок графика;
 
    NameVectorX --- название оси Ox. В формате: [обозначение], [расшифровка]. Например: u, Вероятность выбора;
 
    NameVectorY --- название оси Oy. В формате: [обозначение], [расшифровка]. Например: q, Количество абрикосов;
 
    NameLine1 --- название первого графика (для легенды);
 
    NameLine2 --- название второго графика (для легенды);
 
    Label --- label для графика;
 
    ShowLine --- показывать ли линию;
 
    ShowPoints --- показывать ли точки;
 
    ShowArea --- показывать ли закрашенную область под кривой;
 
    ShowSpecPoints --- показывать ли специальные точки;
 
    ForNormalSize --- нормальный размер графика (на всю ширину) или для маленького размера график создается;
 
    GrayStyle --- второй график рисовать серым, а не красным.
	
\textbf{Возвращаемое значение:}

Строка с Latex кодами с выводимым графиком.


\subsubsection{THQt\_LatexShowTwoIndependentChartsOfPointsAndLine}\label{THQt_LatexShowTwoIndependentChartsOfPointsAndLine}

Функция возвращает строку с выводом некоторого двух независимых графиков по точкам с Latex кодами. Один график выводится в виде точек, а второй в виде линии. Удобно для отображения регрессий.


\begin{lstlisting}[label=code_syntax_THQt_LatexShowTwoIndependentChartsOfPointsAndLine,caption=Синтаксис]
template <class T> QString THQt_LatexShowTwoIndependentChartsOfPointsAndLine (T *VMHL_VectorX1,T *VMHL_VectorY1,int VMHL_N1,T *VMHL_VectorX2,T *VMHL_VectorY2, int VMHL_N2, QString TitleChart, QString NameVectorX, QString NameVectorY,QString NameLine1, QString NameLine2, QString Label,bool ShowLine,bool ShowPoints,bool ShowArea,bool ShowSpecPoints, bool ForNormalSize, bool GrayStyle);
template <class T> QString THQt_LatexShowTwoIndependentChartsOfPointsAndLine (T *VMHL_VectorX1,T *VMHL_VectorY1,int VMHL_N1,T *VMHL_VectorX2,T *VMHL_VectorY2, int VMHL_N2, QString TitleChart, QString NameVectorX, QString NameVectorY,QString NameLine1, QString NameLine2, QString Label,bool ShowLine,bool ShowPoints,bool ShowArea,bool ShowSpecPoints, bool ForNormalSize);
template <class T> QString THQt_LatexShowTwoIndependentChartsOfPointsAndLine (T *VMHL_VectorX1,T *VMHL_VectorY1,int VMHL_N1,T *VMHL_VectorX2,T *VMHL_VectorY2, int VMHL_N2, QString TitleChart, QString NameVectorX, QString NameVectorY,QString NameLine1, QString NameLine2, QString Label,bool ShowLine,bool ShowPoints,bool ShowArea,bool ShowSpecPoints);
template <class T> QString THQt_LatexShowTwoIndependentChartsOfPointsAndLine (T *VMHL_VectorX1,T *VMHL_VectorY1,int VMHL_N1,T *VMHL_VectorX2,T *VMHL_VectorY2, int VMHL_N2, QString TitleChart, QString NameVectorX, QString NameVectorY,QString NameLine1, QString NameLine2, QString Label);
template <class T> QString THQt_LatexShowTwoIndependentChartsOfPointsAndLine (T *VMHL_VectorX1,T *VMHL_VectorY1,int VMHL_N1,T *VMHL_VectorX2,T *VMHL_VectorY2, int VMHL_N2);
\end{lstlisting}

\textbf{Входные параметры:}
 
    VMHL\_VectorX1 --- указатель на вектор координат X точек первой линии;
 
    VMHL\_VectorY1 --- указатель на вектор координат Y точек первой линии;
 
    VMHL\_N1 --- количество точек первой линии;
 
    VMHL\_VectorX2 --- указатель на вектор координат X точек второй линии;
 
    VMHL\_VectorY2 --- указатель на вектор координат Y точек второй линии;
 
    VMHL\_N2 --- количество точек второй линии;
 
    TitleChart --- заголовок графика;
 
    NameVectorX --- название оси Ox. В формате: [обозначение], [расшифровка]. Например: u, Вероятность выбора;
 
    NameVectorY --- название оси Oy. В формате: [обозначение], [расшифровка]. Например: q, Количество абрикосов;
 
    NameLine1 --- название первого графика (для легенды);
 
    NameLine2 --- название второго графика (для легенды);
 
    Label --- label для графика;
 
    ShowLine --- показывать ли линию;
 
    ShowPoints --- показывать ли точки;
 
    ShowArea --- показывать ли закрашенную область под кривой;
 
    ShowSpecPoints --- показывать ли специальные точки;
 
    ForNormalSize --- нормальный размер графика (на всю ширину) или для маленького размера график создается;
 
    GrayStyle --- второй график рисовать серым, а не красным.
	
\textbf{Возвращаемое значение:}

Строка с Latex кодами с выводимым графиком.


\subsection{Обработка текста}

\subsubsection{HQt\_ForcedWordWrap}\label{HQt_ForcedWordWrap}

Функция расставляет принудительные переносы в стиле Latex.


\begin{lstlisting}[label=code_syntax_HQt_ForcedWordWrap,caption=Синтаксис]
QString HQt_ForcedWordWrap(QString S);
\end{lstlisting}

\textbf{Входные параметры:}

S --- разбиваемая строка.

\textbf{Возвращаемое значение:}

 Срока с расставленными принудительно переносами.


\subsubsection{HQt\_LatexGreenText}\label{HQt_LatexGreenText}

Функция возвращает строку с выводом зеленого текста.


\begin{lstlisting}[label=code_syntax_HQt_LatexGreenText,caption=Синтаксис]
QString HQt_LatexGreenText (QString String);
\end{lstlisting}

\textbf{Входные параметры:}

String --- непосредственно выводимая строка.

\textbf{Возвращаемое значение:}

Строка с Latex кодами с зеленым текстом.


\subsubsection{HQt\_LatexRedText}\label{HQt_LatexRedText}

Функция возвращает строку с выводом красного текста.


\begin{lstlisting}[label=code_syntax_HQt_LatexRedText,caption=Синтаксис]
QString HQt_LatexRedText (QString String);
\end{lstlisting}

\textbf{Входные параметры:}

String --- непосредственно выводимая строка.

\textbf{Возвращаемое значение:}

Строка с Latex кодами с красным текстом.


\subsubsection{HQt\_TextForLatexToText}\label{HQt_TextForLatexToText}

Функция обрабатывает строку String из переделки функции HQt\_TextToTextForLatex в нормальную строку. Еще удаляются знаки \$, которые обрамляют формулы.


\begin{lstlisting}[label=code_syntax_HQt_TextForLatexToText,caption=Синтаксис]
QString HQt_TextForLatexToText (QString String);
\end{lstlisting}

\textbf{Входные параметры:}

String --- обрабатываемая строка.

\textbf{Возвращаемое значение:}
 
Обработанная строка.


\subsubsection{HQt\_TextToTextForLatex}\label{HQt_TextToTextForLatex}

Функция переводит текст в текст, который можно добавить в Latex код. В-первую очередь, это экранирование некоторых элементов.


\begin{lstlisting}[label=code_syntax_HQt_TextToTextForLatex,caption=Синтаксис]
QString HQt_TextToTextForLatex (QString Text);
\end{lstlisting}

\textbf{Входные параметры:}

TitleX --- непосредственно выводимая строка.

\textbf{Возвращаемое значение:}

Измененный текст, который можно добавлять в LaTeX.


\subsubsection{THQt\_LatexNumberToText}\label{THQt_LatexNumberToText}

Функция выводит число VMHL\_X в строку Latex, причем число выделено жирным.


\begin{lstlisting}[label=code_syntax_THQt_LatexNumberToText,caption=Синтаксис]
template <class T> QString THQt_LatexNumberToText (T VMHL_X);
\end{lstlisting}

\textbf{Входные параметры:}

VMHL\_X --- выводимое число.

\textbf{Возвращаемое значение:}

Строка, в которой записано число.


\subsection{Показ математических выражений}

\subsubsection{THQt\_LatexShowMatrix}\label{THQt_LatexShowMatrix}

Функция возвращает строку с выводом некоторой матрицы VMHL\_Matrix с Latex кодами.


\begin{lstlisting}[label=code_syntax_THQt_LatexShowMatrix,caption=Синтаксис]
template <class T> QString THQt_LatexShowMatrix (T *VMHL_Matrix, int VMHL_N, int VMHL_M, QString TitleMatrix, QString NameMatrix);
template <class T> QString THQt_LatexShowMatrix (T *VMHL_Matrix, int VMHL_N, int VMHL_M, QString NameMatrix);
template <class T> QString THQt_LatexShowMatrix (T *VMHL_Matrix, int VMHL_N, int VMHL_M);
\end{lstlisting}

\textbf{Входные параметры:}

VMHL\_Matrix --- указатель на выводимую матрицу;
 
VMHL\_N --- количество строк в матрице;
 
VMHL\_M --- количество столбцов в матрице;
 
TitleMatrix --- заголовок выводимой матрицы;
 
NameMatrix --- обозначение матрицы.
	
\textbf{Возвращаемое значение:}

Строка с Latex кодами с выводимой матрицей.


\subsubsection{THQt\_LatexShowMatrix2}\label{THQt_LatexShowMatrix2}

Функция возвращает строку с выводом некоторой матрицы VMHL\_Matrix с Latex кодами.


\begin{lstlisting}[label=code_syntax_THQt_LatexShowMatrix2,caption=Синтаксис]
QString THQt_LatexShowMatrix (QStringList *VMHL_Matrix, int VMHL_N, QString TitleMatrix, QString NameMatrix);
QString THQt_LatexShowMatrix (QStringList *VMHL_Matrix, int VMHL_N, QString NameMatrix);
QString THQt_LatexShowMatrix (QStringList *VMHL_Matrix, int VMHL_N);
\end{lstlisting}

\textbf{Входные параметры:}

    VMHL\_Matrix --- указатель на выводимую матрицу;
 
    VMHL\_N --- количество строк в матрице;
 
    TitleMatrix --- заголовок выводимой матрицы;
 
    NameMatrix --- обозначение матрицы.
	
\textbf{Возвращаемое значение:}

Строка с Latex кодами с выводимой матрицей.


\subsubsection{THQt\_LatexShowVector}\label{THQt_LatexShowVector}

Функция возвращает строку с выводом некоторого вектора VMHL\_Vector с Latex кодами.


\begin{lstlisting}[label=code_syntax_THQt_LatexShowVector,caption=Синтаксис]
template <class T> QString THQt_LatexShowVector (T *VMHL_Vector, int VMHL_N, QString TitleVector, QString NameVector);
template <class T> QString THQt_LatexShowVector (T *VMHL_Vector, int VMHL_N, QString NameVector);
template <class T> QString THQt_LatexShowVector (T *VMHL_Vector, int VMHL_N);
\end{lstlisting}

\textbf{Входные параметры:}
 
    VMHL\_Vector --- указатель на выводимый вектор;
 
    VMHL\_N --- количество элементов вектора;
 
    TitleVector --- заголовок выводимого вектора;
 
    NameVector --- обозначение вектора.
	
\textbf{Возвращаемое значение:}

Строка с Latex кодами с выводимым вектором.


\subsubsection{THQt\_LatexShowVector2}\label{THQt_LatexShowVector2}

Функция возвращает строку с выводом некоторого вектора VMH\_Vector с Latex кодами.


\begin{lstlisting}[label=code_syntax_THQt_LatexShowVector2,caption=Синтаксис]
QString THQt_LatexShowVector (QStringList VMHL_Vector, QString TitleVector, QString NameVector);
QString THQt_LatexShowVector (QStringList VMHL_Vector, QString NameVector);
QString THQt_LatexShowVector (QStringList VMHL_Vector);
\end{lstlisting}

\textbf{Входные параметры:}
 
    VMHL\_Vector --- указатель на выводимый вектор;
 
    TitleVector --- заголовок выводимого вектора;
 
    NameVector --- обозначение вектора.
	
\textbf{Возвращаемое значение:}

Строка с Latex кодами с выводимым вектором.


\subsubsection{THQt\_LatexShowVectorT}\label{THQt_LatexShowVectorT}

Функция возвращает строку с выводом некоторого вектора VMHL\_Vector в транспонированном виде с Latex кодами.


\begin{lstlisting}[label=code_syntax_THQt_LatexShowVectorT,caption=Синтаксис]
template <class T> QString THQt_LatexShowVectorT (T *VMHL_Vector, int VMHL_N, QString TitleVector, QString NameVector);
template <class T> QString THQt_LatexShowVectorT (T *VMHL_Vector, int VMHL_N, QString NameVector);
template <class T> QString THQt_LatexShowVectorT (T *VMHL_Vector, int VMHL_N);
\end{lstlisting}

\textbf{Входные параметры:}
 
    VMHL\_Vector --- указатель на выводимый вектор;
 
    VMHL\_N --- количество элементов вектора;
 
    TitleVector --- заголовок выводимого вектора;
 
    NameVector --- обозначение вектора.
	
\textbf{Возвращаемое значение:}

Строка с Latex кодами с выводимым вектором.


\subsection{Составные изображения}

\subsubsection{HQt\_LatexBeginCompositionFigure}\label{HQt_LatexBeginCompositionFigure}

Функция возвращает строку с выводом начала рисунка, состоящего из нескольких рисунков или графиков.


\begin{lstlisting}[label=code_syntax_HQt_LatexBeginCompositionFigure,caption=Синтаксис]
QString HQt_LatexBeginCompositionFigure ();
\end{lstlisting}

\textbf{Входные параметры:}

Отсутствует.

\textbf{Возвращаемое значение:}

Строка с Latex кодами.


\begin{lstlisting}[label=code_use_HQt_LatexBeginCompositionFigure,caption=Пример использования]
Latex += HQt_LatexBeginCompositionFigure ();
Latex += HQt_LatexBeginFigureInCompositionFigure ();
Latex += THQt_LatexShowChartOfLine (dataX,dataY,N,"Тестовый график","u, Вероятность выбора","q, Количество воронов","линия","plot1",true,true,true,true,false,false);
Latex += HQt_LatexEndFigureInCompositionFigure ();
Latex += HQt_LatexBeginFigureInCompositionFigure ();
Latex += THQt_LatexShowChartOfLine (dataX,dataY,N,"Тестовый график","u, Вероятность выбора","q, Количество воронов","линия","plot2",true,true,true,true,false,false);
Latex += HQt_LatexEndFigureInCompositionFigure ();
Latex += HQt_LatexEndCompositionFigure ("Два графика", "TwoFig");
\end{lstlisting}

\subsubsection{HQt\_LatexBeginFigureInCompositionFigure}\label{HQt_LatexBeginFigureInCompositionFigure}

Функция возвращает строку с Latex кодом при добавлении дополнительного рисунка или графика в рисунок, состоящего из нескольких рисунков.


\begin{lstlisting}[label=code_syntax_HQt_LatexBeginFigureInCompositionFigure,caption=Синтаксис]
QString HQt_LatexBeginFigureInCompositionFigure ();
\end{lstlisting}

\textbf{Входные параметры:}

Отсутствует.

\textbf{Возвращаемое значение:}

Строка с Latex кодами.


\subsubsection{HQt\_LatexEndCompositionFigure}\label{HQt_LatexEndCompositionFigure}

Функция возвращает строку с выводом окончания рисунка, состоящего из нескольких рисунков или графиков.


\begin{lstlisting}[label=code_syntax_HQt_LatexEndCompositionFigure,caption=Синтаксис]
QString HQt_LatexEndCompositionFigure (QString TitleFigure, QString Label);
QString HQt_LatexEndCompositionFigure (QString TitleFigure);
QString HQt_LatexEndCompositionFigure ();
\end{lstlisting}

\textbf{Входные параметры:}

TitleFigure --- заголовок рисунка;

     Label --- label для рисунка.

\textbf{Возвращаемое значение:}

Строка с Latex кодами.


\begin{lstlisting}[label=code_use_HQt_LatexEndCompositionFigure,caption=Пример использования]
Latex += HQt_LatexBeginCompositionFigure ();
Latex += HQt_LatexBeginFigureInCompositionFigure ();
Latex += THQt_LatexShowChartOfLine (dataX,dataY,N,"Тестовый график","u, Вероятность выбора","q, Количество воронов","линия","plot1",true,true,true,true,false,false);
Latex += HQt_LatexEndFigureInCompositionFigure ();
Latex += HQt_LatexBeginFigureInCompositionFigure ();
Latex += THQt_LatexShowChartOfLine (dataX,dataY,N,"Тестовый график","u, Вероятность выбора","q, Количество воронов","линия","plot2",true,true,true,true,false,false);
Latex += HQt_LatexEndFigureInCompositionFigure ();
Latex += HQt_LatexEndCompositionFigure ("Два графика", "TwoFig");
\end{lstlisting}

\subsubsection{HQt\_LatexEndFigureInCompositionFigure}\label{HQt_LatexEndFigureInCompositionFigure}

Функция возвращает строку с Latex кодом после добавлении дополнительного рисунка или графика в рисунок, состоящего из нескольких рисунков.


\begin{lstlisting}[label=code_syntax_HQt_LatexEndFigureInCompositionFigure,caption=Синтаксис]
QString HQt_LatexEndFigureInCompositionFigure ();
\end{lstlisting}

\textbf{Входные параметры:}

Отсутствует.

\textbf{Возвращаемое значение:}

Строка с Latex кодами.


\subsection{Таблицы}

\subsubsection{HQt\_LatexShowTable}\label{HQt_LatexShowTable}

Функция возвращает строку с выводом таблицы.


\begin{lstlisting}[label=code_syntax_HQt_LatexShowTable,caption=Синтаксис]
QString HQt_LatexShowTable (QStringList Col1, QStringList Col2, QString NameCol1, QString NameCol2, double WidthCol1, QString Title);
QString HQt_LatexShowTable (QStringList Col1, QStringList Col2, QStringList Col3, QString NameCol1, QString NameCol2, QString NameCol3, double WidthCol1, double WidthCol2, QString Title);
\end{lstlisting}

\textbf{Входные параметры:}
 
    Col1 --- список строк первого столбца;
 
    Col2 --- список строк второго столбца;
 
    NameCol1--- заголовок первого столбца;
 
    NameCol2--- заголовок второго столбца;
 
    WidthCol1 --- ширина первого столбца в процентах, например 50%;
 
    Title --- заголовок таблицы.

\textbf{Возвращаемое значение:}

Строка с Latex кодами с выводимой таблицы.


\subsection{Текст}

\subsubsection{HQt\_LatexShowAlert}\label{HQt_LatexShowAlert}

Функция возвращает строку с выводом некоторого предупреждения.


\begin{lstlisting}[label=code_syntax_HQt_LatexShowAlert,caption=Синтаксис]
QString HQt_LatexShowAlert (QString String);
\end{lstlisting}

\textbf{Входные параметры:}

String --- непосредственно выводимая строка.

\textbf{Возвращаемое значение:}

Строка с Latex кодами с выводимым предупреждением.


\subsubsection{HQt\_LatexShowHr}\label{HQt_LatexShowHr}

Функция возвращает строку с выводом горизонтальной линии.


\begin{lstlisting}[label=code_syntax_HQt_LatexShowHr,caption=Синтаксис]
QString HQt_LatexShowHr ();
\end{lstlisting}

\textbf{Входные параметры:}

Отсутствуют.

\textbf{Возвращаемое значение:}

Строка с Latex кодами с тэгом горизонтальной линии.


\subsubsection{HQt\_LatexShowSection}\label{HQt_LatexShowSection}

Функция возвращает строку с выводом некоторой строки в виде заголовка.


\begin{lstlisting}[label=code_syntax_HQt_LatexShowSection,caption=Синтаксис]
QString HQt_LatexShowSection (QString String);
\end{lstlisting}

\textbf{Входные параметры:}

String --- непосредственно выводимая строка.

\textbf{Возвращаемое значение:}

Строка с Latex кодами с выводимой строкой.


\subsubsection{HQt\_LatexShowSimpleText}\label{HQt_LatexShowSimpleText}

Функция возвращает строку с выводом некоторой строки с Latex кодами без всякого излишества.


\begin{lstlisting}[label=code_syntax_HQt_LatexShowSimpleText,caption=Синтаксис]
QString HQt_LatexShowSimpleText (QString String);
\end{lstlisting}

\textbf{Входные параметры:}

String --- непосредственно выводимая строка.

\textbf{Возвращаемое значение:}

Строка с Latex кодами с выводимой строкой.


\subsubsection{HQt\_LatexShowSubsection}\label{HQt_LatexShowSubsection}

Функция возвращает строку с выводом некоторой строки в виде подзаголовка.


\begin{lstlisting}[label=code_syntax_HQt_LatexShowSubsection,caption=Синтаксис]
QString HQt_LatexShowSubsection (QString String);
\end{lstlisting}

\textbf{Входные параметры:}

String --- непосредственно выводимая строка.

\textbf{Возвращаемое значение:}

Строка с Latex кодами с выводимой строкой.


\subsubsection{HQt\_LatexShowText}\label{HQt_LatexShowText}

Функция возвращает строку с выводом некоторой строки с Latex кодами.


\begin{lstlisting}[label=code_syntax_HQt_LatexShowText,caption=Синтаксис]
QString HQt_LatexShowText (QString TitleX);
\end{lstlisting}

\textbf{Входные параметры:}

TitleX --- непосредственно выводимая строка.

\textbf{Возвращаемое значение:}

Строка с Latex кодами с выводимой строкой (в виде абзаца).


\subsubsection{THQt\_LatexShowNumber}\label{THQt_LatexShowNumber}

Функция возвращает строку с выводом некоторого числа VMHL\_X с Latex кодами.


\begin{lstlisting}[label=code_syntax_THQt_LatexShowNumber,caption=Синтаксис]
template <class T> QString THQt_LatexShowNumber (T VMHL_X, QString TitleX, QString NameX);
template <class T> QString THQt_LatexShowNumber (T VMHL_X, QString NameX);
template <class T> QString THQt_LatexShowNumber (T VMHL_X);
\end{lstlisting}

\textbf{Входные параметры:}
 
VMHL\_X --- выводимое число;
 
TitleX --- заголовок выводимого числа;
 
NameX --- обозначение числа.

\textbf{Возвращаемое значение:}

Строка с Latex кодами с выводимым числом.

%%%%%%%%%%%%%%%%%%%%%%%%%%%%%%%%%%%%%%%%%%%%%%%%%%%%%%%%%%

\end{document}