\textbf{Входные параметры:}

Left\_X --- левая граница по оси Ox;
 
    Right\_X --- правая граница по оси Ox;
 
    Left\_Y --- левая граница по оси Oy;
 
    Right\_Y --- правая граница по оси Oy;
 
    N --- сколько нужно построить точек по каждой оси. В итоге получим N*N точек;
 
    Function --- ссылка на отрисовываемую двумерную функцию;
 
    TitleChart --- заголовок графика;
 
    NameVectorX --- название оси Ox. В формате: [обозначение], [расшифровка]. Например: u, Вероятность выбора;
 
    NameVectorY --- название оси Oy. В формате: [обозначение], [расшифровка]. Например: q, Количество абрикосов;
 
    NameVectorZ --- название оси Oz. В формате: [обозначение], [расшифровка]. Например: z, Вероятность;
 
    Label --- label для графика;
 
    ColorMap --- какой раскраски будет график. Возможны значения: mathcad, matlab, hot или тот, что вы хотите использовать. Рекомендуется mathcad;
 
    Type --- тип графика. Возможные значения:
 
       Plot3D\_Points --- в виде точек,
 
       Plot3D\_Surface --- в виде поверхности с непрерывной заливкой,
 
       Plot3D\_SurfaceGrid --- в виде поверхности с сеточной заливкой,
 
       Plot3D\_TopView --- вид сверху;
 
    Opacity --- прозрачность графика. Может изменяться от 0 до 1;
 
    AngleHorizontal --- угол поворота графика по горизонтали в градусах от ---180 до 180. Рекомендуется 25;
 
    AngleVertical --- угол поворота графика по вертикали в градусах от ---180 до 180. Рекомендуется 30;
 
    ColorBar --- рисовать с графиком колонку с градациями цветов или нет;
 
    ForNormalSize --- нормальный размер графика (на всю ширину), или для маленького размера график создается.

\textbf{Возвращаемое значение:}

Строка с Latex кодом.