\textbf{Входные параметры:}
 
VHQt\_MatrixXY --- указатель на матрицу значений X и Y графиков;
 
VHQt\_N\_EveryCol --- количество элементов в каждом столбце (так как столбцы идут по парам, то число элементов в нечетном и
 
следующем за ним четном столбце должны совпадать, например 10,10,5,5,7,7);
 
VHQt\_M --- количество столбцов матрицы (должно быть четным числом конечно);
 
TitleChart --- заголовок графика;
 
NameVectorX --- название оси Ox. В формате: [обозначение], [расшифровка]. Например: u, Вероятность выбора;
 
NameVectorY --- название оси Oy. В формате: [обозначение], [расшифровка]. Например: q, Количество абрикосов;
 
NameLine --- указатель на вектор названий графиков (для легенды) количество элементов VHQt\_M/2;
 
Label --- label для графика;
 
ShowLine --- показывать ли линию;
 
ShowPoints --- показывать ли точки;
 
ShowArea --- показывать ли закрашенную область под кривой;
 
ShowSpecPoints --- показывать ли специальные точки;
 
ForNormalSize --- нормальный размер графика (на всю ширину) или для маленького размера график создается;
 
GrayStyle --- серый стиль графиков;
 
SolidStyle --- линии делать сплошными или разными по типу (точками, тире и др.);
 
CircleStyle --- точки все делать кругляшками или нет.
	
\textbf{Возвращаемое значение:}

Строка с Latex кодами с выводимым графиком.