\textbf{Входные параметры:}
 
VMHL\_VectorX1 --- указатель на вектор координат X точек первой линии;
 
VMHL\_VectorY1 --- указатель на вектор координат Y точек первой линии;
 
VMHL\_N1 --- количество точек первой линии;
 
VMHL\_VectorX2 --- указатель на вектор координат X точек второй линии;
 
VMHL\_VectorY2 --- указатель на вектор координат Y точек второй линии;
 
VMHL\_N2 --- количество точек второй линии;
 
TitleChart --- заголовок графика;
 
NameVectorX --- название оси Ox. В формате: [обозначение], [расшифровка]. Например: u, Вероятность выбора;
 
NameVectorY --- название оси Oy. В формате: [обозначение], [расшифровка]. Например: q, Количество абрикосов;
 
NameLine1 --- название первого графика (для легенды);
 
NameLine2 --- название второго графика (для легенды);
 
Label --- label для графика;
 
ShowLine --- показывать ли линию;
 
ShowPoints --- показывать ли точки;
 
ShowArea --- показывать ли закрашенную область под кривой;
 
ShowSpecPoints --- показывать ли специальные точки;
 
ForNormalSize --- нормальный размер графика (на всю ширину) или для маленького размера график создается;
 
GrayStyle --- второй график рисовать серым, а не красным.
	
\textbf{Возвращаемое значение:}

Строка с Latex кодами с выводимым графиком.