\textbf{Входные параметры:}
 
VMHL\_MatrixXY --- указатель на матрицу значений X и Y графиков;
 
    VMHL\_N --- количество точек;
 
    VMHL\_M --- количество столбцов матрицы (1+количество графиков);
 
    TitleChart --- заголовок графика;
 
    NameVectorX --- название оси Ox. В формате: [обозначение], [расшифровка]. Например: u, Вероятность выбора;;
 
    NameVectorY --- название оси Oy. В формате: [обозначение], [расшифровка]. Например: q, Количество абрикосов;
 
    NameLine --- указатель на вектор названий графиков (для легенды) количество элементов VMHL\_M---1 (так как первый столбец --- это X значения);
 
    Label --- label для графика;
 
    ShowLine --- показывать ли линию;
 
    ShowPoints --- показывать ли точки;
 
    ShowArea --- показывать ли закрашенную область под кривой;
 
    ShowSpecPoints --- показывать ли специальные точки;
 
    ForNormalSize --- нормальный размер графика (на всю ширину) или для маленького размера график создается;
 
    GrayStyle --- серый стиль графиков;
 
    SolidStyle --- линии делать сплошными или разными по типу (точками, тире и др.);
 
    CircleStyle --- точки все делать кругляшками или нет.
	
\textbf{Возвращаемое значение:}

Строка с Latex кодами с выводимым графиком.