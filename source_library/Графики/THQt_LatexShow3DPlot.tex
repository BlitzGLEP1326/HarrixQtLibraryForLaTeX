\textbf{Входные параметры:}
 
VMHL\_VectorX --- указатель на вектор значений координат X сетки точек. Количество элементов VMHL\_N;
 
VMHL\_VectorY --- указатель на вектор значений координат Y сетки точек. Количество элементов VMHL\_M;
 
VMHL\_VectorZ --- указатель на матрицу значений координат Z точек. Количество элементов VMHL\_NxVMHL\_M;
 
VMHL\_N --- количество значений в сетке по оси Ox;
 
VMHL\_M --- количество значений в сетке по оси Oy;
 
TitleChart --- заголовок графика;
 
NameVectorX --- название оси Ox. В формате: [обозначение], [расшифровка]. Например: u, Вероятность выбора;
 
NameVectorY --- название оси Oy. В формате: [обозначение], [расшифровка]. Например: q, Количество абрикосов;
 
NameVectorZ --- название оси Oz. В формате: [обозначение], [расшифровка]. Например: z, Вероятность;
 
Label --- label для графика;
 
ColorMap --- какой раскраски будет график. Возможны значения: mathcad, matlab, hot или тот, что вы хотите использовать. Рекомендуется mathcad;
 
Type --- тип графика. Возможные значения:

\begin{itemize}
\item Plot3D\_Points --- в виде точек,
 
\item Plot3D\_Surface --- в виде поверхности с непрерывной заливкой,
 
\item Plot3D\_SurfaceGrid --- в виде поверхности с сеточной заливкой,
 
\item Plot3D\_TopView --- вид сверху;
\end{itemize}
 
Opacity --- прозрачность графика. Может изменяться от 0 до 1;
 
AngleHorizontal --- угол поворота графика по горизонтали в градусах от ---180 до 180. Рекомендуется 25;
 
AngleVertical --- угол поворота графика по вертикали в градусах от ---180 до 180. Рекомендуется 30;
 
ColorBar --- рисовать с графиком колонку с градациями цветов или нет;
 
ForNormalSize --- нормальный размер графика (на всю ширину), или для маленького размера график создается.
	
\textbf{Возвращаемое значение:}

Строка с Latex кодами с выводимым графиком.